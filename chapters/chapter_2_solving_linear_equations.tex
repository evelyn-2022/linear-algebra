\noindent A systematic way to solve the equation $Ax=b$:
\begin{enumerate}
    \item Apply elimination to $Ax=b$
    \item Get upper triangular $U$
    \item Solve $Ux=c$ by back substitution
\end{enumerate}

\section{The number of solutions to $Ax=b$}

\paragraph{1. Exactly one solution} $A$ has independent columns. $A$ has an inverse matrix $A^{-1}$. Example:

\[
    A =
    \begin{bmatrix}
        2 & 3 \\
        4 & 2
    \end{bmatrix},
    \quad
    b =
    \begin{bmatrix}
        5 \\ 6
    \end{bmatrix}
    \quad\Rightarrow\quad
    x =
    \begin{bmatrix}
        1 \\ 1
    \end{bmatrix}
\]

\paragraph{2. No solution} $b$ is not in the column space of $A$. Example:

\[
    A =
    \begin{bmatrix}
        2 & 3 \\
        4 & 6
    \end{bmatrix},
    \quad
    b =
    \begin{bmatrix}
        6 \\ 15
    \end{bmatrix}
    \quad\Rightarrow\quad
    0=3
\]

\paragraph{3. Infinitely many solutions} $A$ has dependent columns. Example:

\[
    A=
    \begin{bmatrix}
        2 & 3 \\
        4 & 6
    \end{bmatrix},
    \quad
    b=
    \begin{bmatrix}
        6 \\ 12
    \end{bmatrix}
    \quad\Rightarrow\quad
    x=
    \begin{bmatrix}
        3\alpha \\ 6-2\alpha
    \end{bmatrix}
    \text{for any number} \; \alpha
\]

\section{Elimination Matrix $E_{ij}$}

We move column by column from left to right. Typically, the first non-zero element in each column is chosen as the pivot (one row below the row of the pivot we just used).
The pivots are used to eliminate the elements below them.
The elimination matrix $E_{ij}$ eliminates the element $a_{ij}$ in the $i$th row and $j$th column.


\begin{examplex}
    \[
        A =
        \begin{bmatrix}
            2 & 3  & 4  \\
            4 & 11 & 14 \\
            2 & 8  & 17
        \end{bmatrix},
        \quad
        b =
        \begin{bmatrix}
            19 \\ 55 \\ 50
        \end{bmatrix}
    \]

    \paragraph{Step 1}
    The first pivot is $a_{11} = 2$. Eliminate the elements below it. \\
    \vspace{0.4cm}
    Eliminate $a_{21}$ \qquad
    $
        E_{21} =
        \begin{bmatrix}
            1  & 0 & 0 \\
            -2 & 1 & 0 \\
            0  & 0 & 1
        \end{bmatrix}
        \quad
        E_{21}A =
        \begin{bmatrix}
            2 & 3 & 4  \\
            0 & 5 & 6  \\
            2 & 8 & 17
        \end{bmatrix}
        \quad
        E_{21}b =
        \begin{bmatrix}
            19 \\ 17 \\ 50
        \end{bmatrix}
    $

    \vspace{0.4cm}
    Eliminate $a_{31}$ \qquad
    $
        E_{31} =
        \begin{bmatrix}
            1  & 0 & 0 \\
            0  & 1 & 0 \\
            -1 & 0 & 1
        \end{bmatrix}
        \quad
        E_{31}E_{21}A =
        \begin{bmatrix}
            2 & 3 & 4  \\
            0 & 5 & 6  \\
            0 & 5 & 13
        \end{bmatrix}
        \quad
        E_{31}E_{21}b =
        \begin{bmatrix}
            19 \\ 17 \\ 31
        \end{bmatrix}
    $

    \paragraph{Step 2}
    The second pivot is $a_{22} = 5$. Eliminate the elements below it. \\

    \vspace{0.4cm}
    Eliminate $a_{32}$ \qquad
    $
        E_{32} =
        \begin{bmatrix}
            1 & 0  & 0 \\
            0 & 1  & 0 \\
            0 & -1 & 1
        \end{bmatrix}
        \quad
        E_{32}E_{31}E_{21}A =
        \begin{bmatrix}
            2 & 3 & 4 \\
            0 & 5 & 6 \\
            0 & 0 & 7
        \end{bmatrix}
        \quad
        E_{32}E_{31}E_{21}b =
        \begin{bmatrix}
            19 \\ 17 \\ 7
        \end{bmatrix}
    $

    \paragraph{Step 3}
    Solve $Ux=c$ by back substitution. \\

    \vspace{0.4cm}
    Now we have $U = \begin{bmatrix}
            2 & 3 & 4 \\
            0 & 5 & 6 \\
            0 & 0 & 7
        \end{bmatrix}$
    and $c = \begin{bmatrix}
            19 \\ 17 \\ 7
        \end{bmatrix}$. We can go from the bottom up to solve for $x$.
\end{examplex}

\begin{mdframed}
    \textbf{An easy way to come up with elimination matrix} \\

    \noindent To go from $A =
        \begin{bmatrix}
            2 & 3 & 4  \\
            0 & 5 & 6  \\
            2 & 8 & 17
        \end{bmatrix}$ to $B =
        \begin{bmatrix}
            2 & 3 & 4  \\
            0 & 5 & 6  \\
            0 & 5 & 13
        \end{bmatrix}$, we need to come up with $E_{31}$: \\

    \baselineskip=1.5\baselineskip
    \noindent \textbf{1. For the first row in $E_{31}$:} \\
    $A_{row_{1}}$ is $\begin{bmatrix}
            2 & 3 & 4
        \end{bmatrix}$, $B_{row{1}}$ is $\begin{bmatrix}
            2 & 3 & 4
        \end{bmatrix}$. We take $1 \times A_{row_{1}}$, $0 \times A_{row_{2}}$, and $0 \times A_{row_{3}}$. So the first row of $E_{31}$ is $\begin{bmatrix}
            1 & 0 & 0
        \end{bmatrix}$. \\
    \noindent \textbf{2. For the second row in $E_{31}$:} \\
    $A_{row_{2}}$ is $\begin{bmatrix}
            0 & 5 & 6
        \end{bmatrix}$, $B_{row{2}}$ is $\begin{bmatrix}
            0 & 5 & 6
        \end{bmatrix}$. We take $0 \times A_{row_{1}}$, $1 \times A_{row_{2}}$, and $0 \times A_{row_{3}}$. So the second row of $E_{31}$ is $\begin{bmatrix}
            0 & 1 & 0
        \end{bmatrix}$. \\
    \noindent \textbf{3. For the third row in $E_{31}$:} \\
    $A_{row_{3}}$ is $\begin{bmatrix}
            2 & 8 & 17
        \end{bmatrix}$, $B_{row{3}}$ is $\begin{bmatrix}
            0 & 5 & 13
        \end{bmatrix}$. We take $-1 \times A_{row_{1}}$, $0 \times A_{row_{2}}$, and $1 \times A_{row_{3}}$. So the third row of $E_{31}$ is $\begin{bmatrix}
            -1 & 0 & 1
        \end{bmatrix}$. \\
    \noindent \textbf{4. Put it togeter:}
    \[
        E_{31} =
        \begin{bmatrix}
            1  & 0 & 0 \\
            0  & 1 & 0 \\
            -1 & 0 & 1
        \end{bmatrix}
    \]
\end{mdframed}

\section{Permutation Matrix $P$}
