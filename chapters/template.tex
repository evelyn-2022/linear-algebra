\section{Vectors and Linear Combination}
\label{sec:vectors_and_linear_combination}

% Introduction or overview
Linear equations represent geometric lines and planes in space. Understanding the geometry helps in interpreting solutions.

% Key topics in the lecture
\subsection{Linear Equations}
A system of linear equations is:
\[
    a_1x_1 + a_2x_2 + \cdots + a_nx_n = b
\]
where $a_1, a_2, \dots, a_n$ are constants and $x_1, x_2, \dots, x_n$ are variables.

% Examples and Diagrams
\begin{example}
    Consider the system of linear equations:
    \[
        \begin{aligned}
            2x_1 + 3x_2 & = 5  \\
            x_1 - 4x_2  & = -1
        \end{aligned}
    \]
    Graphically, these represent two lines in the plane.
\end{example}

% Diagrams (optional)
\begin{figure}[h]
    \centering
    \caption{Graph of two linear equations}
    \label{fig:geometry_example}
\end{figure}

% Key concepts and formulas
\subsection{Solving Linear Equations}
To solve a system of linear equations, we can use methods such as substitution or elimination.

% Text boxes or callouts
\begin{mdframed}
    \textbf{Term:} This is the explanation for the term.
\end{mdframed}

\begin{tcolorbox}[colframe=blue!75!black, colback=blue!10!white, title=Callout]
    \textbf{Term:} This is the explanation for the term.
\end{tcolorbox}

\shadowbox{\textbf{Term:} This is the explanation for the term.}

\ovalbox{\textbf{Term:} This is the explanation for the term.}

% Conclusion or wrap-up
This lecture introduces the geometric interpretation of linear equations. In the next lecture, we'll dive deeper into vector spaces.

% Label references for easy navigation
\label{sec:vectors_and_linear_combination_end}
